\documentclass{article}

\title{SameGame más nevén Cilckomania NP teljes puzzle megoldásának megközelítése különböző módszerekkel}
\author{Kárpáti Márk András}
\date{2022 Tavaszi félév}

\begin{document}
	
	\maketitle
	
	Önálló laboratórium
	
	%Logic and Simulation of Interaction and Reasoning (AISB 2008 Proceedings volume 9)
	%Addressing NP-Complete Puzzles with Monte-Carlo Methods page 55
	%Maarten P.D. Schadd, Mark H.M. Winands, H. Jaap van den Herik & Huib Aldewereld
	
	Használt Szabályok:
	\begin{itemize}		
		\item  Minden mezőre kiválasztható és ér pontot.
		\item Ha egy mezőt kiválasztunk az eltűnik, ha egy mező eltűnik, akkor az összes vele szomszédos azonos azonosítójú/színű mező is eltűnik.
		\item Egy lépés \( (n-1)^2 \) pontot ér, ahol n az egyszerre eltüntet mezők száma
		\item Nincs idő korlát.
		\item Nincs lépés szám korlát.
		\item Minden lépés után a játék pálya "összeomlik" ha van üres hely az elemek lefelé mozdulnak, ha van egy egész üres oszlop, akkor balra felé mozdulnak.
	\end{itemize}
	
	Főbb funkciói a programnak:
	\begin{itemize}
		\item Teljesen véletlenszerű pálya generálás, ahol a pálya mérete és a mezők típusának számossága megadható
		\item Pálya beolvasása megadott .csv formátumban.
		\item Pálya kiírása a beolvasáshoz megfelelő .csv formátumban.		
	\end{itemize}
	
	A pálya .csv formátumú fájlokban van tárolva, aminek első sora oszlop és sorok száma. Ezt követi a pálya az összes adat ';' karakterrel van elválasztva.
	
	%Komment sor, csak hogy tudjam, hogy % a komment karakter
	
\end{document}