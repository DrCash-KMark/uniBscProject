\documentclass{article}

\title{SameGame NP teljes puzzle megoldásának megközelítése különböző módszerekkel}
\author{Kárpáti Márk András}
\date{2022 Tavaszi félév}

\begin{document}
	
	\maketitle
	
	
	\section{A SameGame játék}
	
	SameGame egy puzzle játék, amit Kuniaki Moribe talált ki chain shot! néven 1985-ben. Fujitsu FM-8/7 sorozatra volt kiadva, és egy havi lapban jelent meg egy számítógépes napi lapban. \cite{ChainShot1985}. Később Eiji Fukumata a SameGame névvel újra kreaálta 1992-ben.
	
	\subsection{Szabályok}
	
	\begin{itemize}		
		\item pálya mérete 15x15
		\item 5 különböző fajta mező szín van.
		\item Kezdetben a pálya összes mezején van valamilyen szín
		\item Egy lépés egy csoport eltávolításából áll.
		\item Egy csoport legalább 2 egyszínű mező, ami érintkezik egymással, szomszédosak.
		\item Egy lépés \( (n-1)^2 \) pontot ér, ahol n az egy lépésben eltüntet mezők száma
		\item Nincs idő korlát.
		\item Nincs lépés szám korlát.
		\item Minden lépés után a játék pálya "összeomlik" ha van üres hely az elemek lefelé mozdulnak, ha van egy egész üres oszlop, akkor balra felé tolódnak az oszlopok.
		\item A játék véget ér, ha elfogytak a mezők vagy nem marad lehetséges lépés
		
	\end{itemize}
	
	\subsection{SamGame komplexitása}
	
	A komplexitása, azt mutatja, hogy mennyire nehéz megoldani a játékot. Két fontos mértéke a játék-fa komplexitás és az állapottér komplexitás.\cite{allis1994searching} A játék-fa kompleszítás a levélcsomópontok száma a megoldási fában az induló pozícióból. Az állapottér kompleszítás az elérhető játék állások a kezdő pozicionál.
	\\	
	\\
	A SameGambe-ben a játék-fa komplexitás nagyjából \(b^d\), ahol b az átlagos elágazási faktor és a d az átlagos játék hossza.
	
	\section{Kereső algoritmusok puzzle-okhoz}	
	
	Klasszikus megközelítése a puzzle-k megoldásának \textbf{A*} \cite{hart1968formal} és az \textbf{IDA*} \cite{korf1985depth}. Az A* egy Legjobbat először keresés, ahol minden csúcs el van tárolva egy listában. Ez a lista rendezve van egy elfogadható kiértékelő függvénnyel. Minden iterációban az első elemet eltávolítjuk, és a gyereke elemeit hozzá adjuk a rendezett listához. Ezt addig ismételjük, amíg a célállapot nem kerül a lista legelejére. IDA* az egy iteratívan mélyülő változata az A* keresésnek. Úgy használ mélységi bejárást, hogy nem kell eltárolni az egész fát a memóriában. Addig keres mélységi eljárással, amíg nem ér levébe és a kiértékelő függvény értéke nem ér el egy küszöböt. Ha a keresés eredmény nélkül tér vissza, akkor a küszöböt növeljük.
	\\
	\\
	Mind két módszer nagyon erősen függ a kiértékelő függvény minőségétől. Még ha a függvény egy elfogadható alsó becslés, akkor is elég pontosnak kell lennie. Klasszikus puzzle játékoknál mint a Nyolcas játék, annak nagyobb változatai \cite{korf1985depth}  és a Sokoban \cite{junghanns2000pushing} ez a megközelítés nagyon jól működik. Hiszen ott egy jó alul-becslő függvény a Manhattan távolság. A fő feladata ennek területnek a kiértékelő függvény javítása pl mintaadatbázisokkal. \cite{felner2005dual} \cite{culberson1998pattern}
	\\
	\\
	Ezek a megközelítési módok elbuknak a SameGame esetén, mert nem annyira egyszerű egy elfogadható függvényt (nem becsli túl az árat) készíteni, ami még mindig elég pontos. Erre az egyik próbálkozás, hogy pontokat asszociálunk a táblán lévő csoportokhoz. Ez vissza is adja azt a pontszámot, amit nagyjából kapnánk, ha föntről lefelé haladva eltávolítanánk ezeket a csoportokat. De ha egy optimális pontszámot szeretnénk elérni a SameGame-ben, akkor a jelenlegi helyzethez egy "túl becslő" lenne szükségünk, hiszen nem a legrövidebb utat keressük, hanem a legtöbb pontot szeretnénk elérni. Egy elfogadható becslő függvényt kapunk, ha úgy tekintjük, hogy az összes ugyan olyan színű mezőt egyszerre el tudjuk távolítani a pályáról. Ezt lehet javítani azzal, hogy mínusz pontot ad, ha ezzel a táblát nem tudjuk teljesen kiüríteni. Azonban ez messze van az igazi ponttól, amit egy helyzetben lehet szerezni, és nem működik jól az A* és IDA*-val sem. A tesztek azt mutatják, hogy ezzel a becslővel ellátott A*, IDA* Szélességi kereséshez hasonlóan viselkednek.\cite{SCHADD20123} A probléma az, hogy a gyerek heurisztikus értéke sokkal kisebb lesz mint a szülője, hacsak nem egy nagy csoportot távolított el. Sejtések szerint ugyan ettől szenvednek a mélységi keresés, az Elágazás és korlátozás algoritmus is. \cite{vempaty1991depth} Ezekből kifolyólag jelent kihívást a SameGame a puzzle-k kutatásában.
	
	
	\section{Megközelítés}
	
	A cél az, hogy minél jobb algoritmust találjak, a játék megoldására. Ehhez először könnyítettem a játék szabályokon. Az így készült algoritmussal és az eredeti szabályokkal legenerált játék pályákról nagy valószínűséggel megmondható lesz, hogy meglehet-e oldani őket.
	
	
	%Logic and Simulation of Interaction and Reasoning (AISB 2008 Proceedings volume 9)
	%Addressing NP-Complete Puzzles with Monte-Carlo Methods page 55
	%Maarten P.D. Schadd, Mark H.M. Winands, H. Jaap van den Herik & Huib Aldewereld
	
	Használt Szabályok:
	
	
	Főbb funkciói a programnak:
	\begin{itemize}
		\item Teljesen véletlenszerű pálya generálás, ahol a pálya mérete és a mezők típusának számossága megadható
		\item Pálya beolvasása megadott .csv formátumban.
		\item Pálya kiírása a beolvasáshoz megfelelő .csv formátumban.		
	\end{itemize}
	
	A pálya .csv formátumú fájlokban van tárolva, aminek első sora oszlop és sorok száma. Ezt követi a pálya az összes adat ';' karakterrel van elválasztva.
	
	%Komment sor, csak hogy tudjam, hogy % a komment karakter
	
	Elmélet:
	
	\bibliography{onlabForrasok}
	\bibliographystyle{plain}
	
\end{document}