\documentclass{article}


\title{SameGame NP teljes puzzle megoldásának megközelítése különböző módszerekkel}
\author{Kárpáti Márk András}
\date{2022 Tavaszi félév}

\begin{document}
	
	\maketitle
	
	
	\section{Játék}
	
	SameGame az egy NP teljes, egyjátékosos, teljesinformációs  játék. Tradicionálisan a játék szabályai: pálya egy tetszőleges téglalap általában 15x15 méretű. 5 különböző színű mező van. Egy lépés abból ál, hogy eltávolítunk egy ortogonálisan illeszkedő elem csoportot. Az elemek lezuhannak, és amint megjelenik egy üres oszlop a jobb oldalon lévő oszlopok balra tolódnak. A játék akkor ér véget, ha elfogytak a mezők. A variációkat ahol a színek száma más vagy a pálya mérete gyakran más néven nevezik pl.: Clickomania.
	
	\section{Klasszikus metódusok A* és a IDA*}
	
	
	
	\section{Megközelítés}
	
	A cél az, hogy minél jobb algoritmust találjak, a játék megoldására. Ehhez először könnyítettem a játék szabályokon. Az így készült algoritmussal és az eredeti szabályokkal legenerált játék pályákról nagy valószínűséggel megmondható lesz, hogy meglehet-e oldani őket.
	
	
	%Logic and Simulation of Interaction and Reasoning (AISB 2008 Proceedings volume 9)
	%Addressing NP-Complete Puzzles with Monte-Carlo Methods page 55
	%Maarten P.D. Schadd, Mark H.M. Winands, H. Jaap van den Herik & Huib Aldewereld
	
	Használt Szabályok:
	\begin{itemize}		
		\item pálya mérete 15x16
		\item 5 különböző fajta mező szín van (én ezt számokkal jelölöm)
		\item  Minden mezőre kiválasztható és ér pontot. %minden csoport egy csoport minimum 2 egyszínű mező
		\item Ha egy csoportot kiválasztunk az eltűnik, egy csoportba tartozik az összes vele szomszédos azonos azonosítójú/színű mező.
		\item Egy lépés \( (n-1)^2 \) pontot ér, ahol n az egyszerre eltüntet mezők száma
		\item Nincs idő korlát.
		\item Nincs lépés szám korlát.
		\item Minden lépés után a játék pálya "összeomlik" ha van üres hely az elemek lefelé mozdulnak, ha van egy egész üres oszlop, akkor balra felé tolódnak az oszlopok.
		\item A játék véget ér, ha elfogytak a mezők % vagy nem marad lehetséges lépés
		
	\end{itemize}
	
	Főbb funkciói a programnak:
	\begin{itemize}
		\item Teljesen véletlenszerű pálya generálás, ahol a pálya mérete és a mezők típusának számossága megadható
		\item Pálya beolvasása megadott .csv formátumban.
		\item Pálya kiírása a beolvasáshoz megfelelő .csv formátumban.		
	\end{itemize}
	
	A pálya .csv formátumú fájlokban van tárolva, aminek első sora oszlop és sorok száma. Ezt követi a pálya az összes adat ';' karakterrel van elválasztva.
	
	%Komment sor, csak hogy tudjam, hogy % a komment karakter
	
	Elmélet:
	
	\bibliography{onlabForrasok}
	\bibliographystyle{plain}
	
\end{document}